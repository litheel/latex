\documentclass[12pt, a4paper, twoside]{article}
%\documentclass[12pt, a4paper, oneside]{article}

\usepackage{graphicx}
\usepackage{url}
\usepackage{amsmath}
\usepackage{amssymb}


\title{Second document}
\author{Cicero, Caesar and Vergil\thanks{funded by the ShareLaTeX team}}
\date{\today}

\begin{document}
	
	\maketitle
	
	
	\begin{abstract}
		Cum M.~Cicero consul Nonis Decembribus senatum in aede
		Iovis Statoris consuleret, quid de iis coniurationis Catilinae
		sociis fieri placeret, qui in custodiam traditi essent, factum
		est, ut duae potissimum sententiae proponerentur, una D.~Silani consulis
		designati, qui morte multandos illos censebat,
		altera C.~Caesaris, qui illos publicatis bonis per municipia
		Italiae distribueudos ac vinculis sempiternis tenendos existimabat. Cum
		autem plures senatores ad C.~Caesaris quam ad
		D.~Silani sententiam inclinare viderentur, M.~Cicero ea, quae
		infra legitur, oratione Silani sententiam commendare studuit.
		\vfill
	\end{abstract}
	
	\section{Introduction}
	
	Video, patres conscripti, in me omnium vestrum ora atque oculos esse
	conversos, video vos non solunn de vestro ac rei publicae, verum etiam,
	si id depulsum sit, de meo periculo esse sollicitos. 
	\begin{itemize}
		\item In Catilinam I
		\item In Catilinam II
		\item In Catilinam III
		\item In Catilinam IV
	\end{itemize}
Est mihi iucunda in
malis et grata in dolore vestra erga me voluntas, sed eam, per deos
inmortales, deponite atque obliti salutis meae de vobis ac de vestris
liberis cogitate %\cite{AbTaRu:54},  \cite{Abl:56}, \cite{Keo:58} and \cite{Pow:85}. 
Mihi si haec condicio consulatus data est, ut omnis
acerbitates, onunis dolores cruciatusque perferrem, feram non solum
fortiter, verum etiam lubenter, dum modo meis laboribus vobis populoque
Romano dignitas salusque pariatur.

	
	
	\section{Section}
	
	Ego sum ille consul, patres
	conscripti, cui non forum, iu quo omnis aequitas continetur, non campus
	consularibus auspiciis consecratus, non curia, summum auxilium 
	omnium 	gentium, non domus, commune perfugium, non lectus ad quietem datus, non
	denique haec sedes honoris [sella curulis] umquam vacua mortis periculo
	atque insidiis fuit. 
	
	Multa meo quodam dolore in vestro timore sanavi. Nunc si hunc exitum
	consulatus mei di inmortales esse voluerunt, ut vos populumque Romanum
	ex caede miserrima, coniuges liberosque vestros virginesque Vestales ex
	acerbissima vexatione, templa atque delubra, hanc pulcherrimam patriam
	omnium nostrum ex foedissima flamma, totam Italiam ex bello et vastitate
	eriperem, quaecumque mihi uni proponetur fortuna, subeatur. Etenim, si
	P.~Lentulus suum nomen inductus a vatibus fatale ad perniciem rei
	publicae fore putavit, cur ego non laeter meum consulatum ad salutem
	populi Romani prope fatalem extitisse?
	
	
	\section{The argument}
	Some words might be appropriate describing (\ref{e1}), if we
	had but time and space enough.
	\begin{equation}
		\frac{\partial F}{\partial
			t}=D\frac{\partial^2 F}{\partial x^2},
		\label{e1}
	\end{equation}
	Quare, patres conscripti, consulite vobis, prospicite patriae,
	conservate vos, coniuges, liberos fortunasque vestras, populi Romani
	nomen salutemque defendite; mihi parcere ac de me cogitare desinite. Nam
	primum debeo sperare omnis deos, qui huic urbi praesident, pro eo mihi,
	ac mereor, relaturos esse gratiam; deinde, si quid obtigerit, aequo
	animo paratoque moriar. Nam neque turpis mors forti viro potest accidere
	neque immatura consulari nec misera sapienti. Nec tamen ego sum ille
	ferreus, qui fratris carissimi atque amantissimi praesentis maerore non
	movear horumque omnium lacrumis, a quibus me circumsessum videtis Neque
	meam mentem non domum saepe revocat exanimata uxor et abiecta metu filia
	et parvulus filius quem mihi videtur amplecti res publica tamquam ob
	sidem consulatus mei, neque ille, qui expectans huius exitum diei stat
	in conspectu meo, gener. Moveo his rebus omnibus, sed in eam partem, uti
	salvi sint vobiscum omnes, etiamsi me vis aliqua oppresserit, potius,
	quam et illi et nos una rei publicae peste pereamus.
	Another equation is (\ref{eq:eq2})
	\begin{equation}
		A=\left[
		\begin{array}{ccc} 
			1 & 2 & 3 \\
			4 & 5 & 6 \\
			7 & 8 & 9
		\end{array} 
		\right]
		\label{eq:eq2}
	\end{equation}
	
	Quare, patres conscripti, incumbite ad salutem rei publicae,
	circumspicite omnes procellas, quae inpendent, nisi providetis. Non
	Ti.~Gracchus, quod iterum tribunus plebis fieri voluit, non
	C.~Gracchus, quod agrarios concitare conatus est, non L.~Saturninus,
	quod C.~Memmium occidit, in discrimen aliquod atque in vestrae
	severitatis iudicium adduciturtenentur ii, qui ad urbis incendium, ad
	vestram omnium caedem, ad Catilinam accipiendum Romae restiterunt,
	tenentur litterae, signa, manus, denique unius cuiusque confessio;
	sollicitantur Allobroges, servitia excitantur, Catilina accersitur; id
	est initum consilium, ut interfectis omnibus nemo ne ad deplorandum
	quidem populi Romani nomen atque ad lamentandam tanti imperii
	calamitatem relinquatur.
	
	Haec omnia indices detulerunt, rei confessi sunt, vos multis iam iudiciis
	iudicavistis, primum quod mihi gratias egistis singu laribus verbis et
	mea virtute atque diligentia perditorum hominum coniurationem patefactam
	esse decrevistis, deinde quod P.~Lentulum se abdicare praetura
	coegistis, tum quod eum et ceteros, de quibus iudicastis, in custodiam
	dandos censuistis, maximeque quod meo nomine supplicationem decrevistis,
	qui honos togato habitus ante me est nemini; postremo hesterno die
	praemia legatis Allobrogum Titoque Volturcio dedistis amplissima. Quae
	sunt omnia eius modi, ut ii, qui in custodiam nominatim dati sunt, sine
	ulla dubitatione a vobis damnati esse videantur.
	

	\section{A section}
	Also containing some text.
	
	
	\section{Epilogue}
	A word or two to conclude,  and this even includes some
	inline maths: $R(x,t)\sim
	t^{-\beta}g(x/t^\alpha)\exp(-|x|/t^\alpha)$
	
	\subsection{Last words}
	Quae cum ita sint, pro imperio, pro exercitu, pro provincia, quam
	neglexi, pro triumpho ceterisque laudis insignibus, quae sunt a me
	propter urbis vestraeque salutis custodiam repudiata, pro clientelis
	hospitiisque provincialibus, quae tamen urbanis opibus non minore labore
	tueor quam comparo, pro his igitur omnibus rebus, pro meis in vos
	singularibus studiis proque hac, quam perspicitis, ad conservandam rem
	publicam diligentia nihil a vobis nisi huius temporis totiusque mei
	consulatus memoriam postulo; quae dunn erit in vestris fixa mentibus,
	tutissimo me muro saeptum esse arbitrabor. Quodsi meam spem vis
	inproborum fefellerit atque superaverit, commendo vobis parvum meum
	filium, cui profecto satis erit praesidii non solum ad salutem, verum
	etiam ad dignitatem, si eius, qui haec omnia suo solius periculo
	conservarit, illum filium esse memineritis.
	
	\subsection{The very last words}
	Quapropter de summa salute
	vestra populique Romani, de vestris coniugibus ac liberis, de aris ac
	focis, de fanis atque templis de totius urbis tectis ac sedibus, de
	imperio ac libertate, de salute Italiae, de universa re publica
	decernite diligenter, ut instituistis, ac fortiter. Habetis eum
	consulem, qui et parere vestris decretis non dubitet et ea %\cite{fortran}, quae
	statueritis, quoad vivet, defendere et per se ipsum praestare possit.
	
	

	
	
\end{document}